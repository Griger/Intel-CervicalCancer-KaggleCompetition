\documentclass[10pt,a4paper]{article}
\usepackage[utf8]{inputenc}
\usepackage{amsmath}
\usepackage{amsfonts}
\usepackage{amssymb}
\usepackage[spanish]{babel}

\usepackage{pdflscape}
\usepackage{afterpage}

\usepackage{float}
\usepackage[table,xcdraw]{xcolor} %para usar tablas con color de fondo en las celdas
\usepackage{hyperref} %para poder poner enlaces
\usepackage{listings} %para insertar código
\usepackage{tikz}%para pintar las redes neuronales
%\usepackage{color} %para poder definir y usar colores
\usepackage{soulutf8} %para hacer los subrayados

\usepackage{sectsty} %cambiar el color del título de las secciones y subsecciones

\usepackage{geometry}


\author{\textbf{Alejandro Casado Quijada \& Gustavo Rivas Gervilla}}
\title{\textcolor{deepblue}{\textbf{Intel Cervical Cancer \\Competición Kaggle}}}
\date{}

%Configurando lstlisting para mostrar código Python con algún 	 de colores (copiado de http://tex.stackexchange.com/questions/83882/how-to-highlight-python-syntax-in-latex-listings-lstinputlistings-command) ------------------------------
% Custom colors
\definecolor{deepblue}{RGB}{12, 17, 104}
\definecolor{pygreen}{RGB}{50, 229, 89}
\definecolor{deepgreen}{rgb}{0,0.5,0}
\definecolor{light-gray}{gray}{0.85}
\definecolor{comment-gray}{gray}{0.65}
\definecolor{light-blue}{rgb}{0.6,1,0.8}
\definecolor{light-yellow}{rgb}{1,1,0.6}

% Default fixed font does not support bold face
\DeclareFixedFont{\ttb}{T1}{txtt}{bx}{n}{8} % for bold
\DeclareFixedFont{\ttm}{T1}{txtt}{m}{n}{8}  % for normal

%Configuración de los listings
\lstset{
	language=Python,
	basicstyle=\ttm,
	otherkeywords={self},             % Add keywords here
	keywordstyle=\ttb\color{deepblue},
	emph={MyClass,__init__},          % Custom highlighting
	emphstyle=\ttb\color{deepred},    % Custom highlighting style
	stringstyle=\color{deepgreen},
	frame=tb,                         % Any extra options here
	showstringspaces=false,            % 
	commentstyle=\ttm\color{comment-gray}, % Custom comment style
}
%--------------------------------------------------------------------------------

\newcommand{\emp}[1]{\sethlcolor{light-yellow}\hl{#1}} %Comando para poner código inline
\newcommand{\code}[1]{\textcolor{pygreen}{\texttt{#1}}} %Comando para poner código inline
\newcommand{\archive}[1]{\sethlcolor{light-blue}\hl{\texttt{#1}}} %Comando para resaltar nombres de archivos
\renewcommand\tablename{Tabla} %Cambiar el nombre de las tablas
\renewcommand\figurename{Figura} %Cambiar el nombre de las tablas
\renewcommand{\contentsname}{Índice} %Cambiar el nombre de la ToC

\usepackage{pdfpages}

\hypersetup{
    colorlinks,
    linkcolor={red!50!black},
    citecolor={blue!50!black},
    urlcolor={blue!80!black}
}

\sectionfont{\color{deepblue}}
\subsectionfont{\color{blue}}

\begin{document}
\pagenumbering{gobble}
\maketitle

\begin{center}
  \textbf{Nombre del equipo: }CaGer\\
  \textbf{Ranking global: } $\sim 225$ (dudoso)\\
  \textbf{Puntuación: }0.81792
\end{center}

\newpage

\tableofcontents

\newpage

\pagenumbering{arabic}
\section{Introducción}

En esta práctica se va a trabajar con el dataset de Intel & MobileODT Cervical Cancer Screening. Este dataset está formado por un cojunto de imágenes de úteros con cáncer cervical. Contiene imágenes de 3 tipos distintos de cáncer, estos tipos son llamados: tipo 1, tipo 2 y tipo 3. Por lo que nos encontramos ante un problema de clasificación de multiclases, a diferecia de la práctica anterior, en la cual solo debíamos predecir si el pasajero moría o sobrevivía. El objetivo es predecir, dada una imagen, que tipo de cáncer presenta.\\


En este trabajo se va a trabajar con el dataset del Titanic, un conjunto de datos en el que se plantea un \textbf{problema de clasificación}, dadas una serie de características de un pasajero, que enumeraremos en la siguiente sección, se tendrá que decidir si el pasajero sobrevivió o no a la catastrofe. En primer lugar, como toma de contacto con el dataset y también para tomar algunas ideas que aplicar al conjunto de datos, se han realizado dos tutoriales con planteamientos similares, ambos realizan un preprocesamiento parecido a los datos y emplean Random Forest como algoritmo final de clasificación, después de probar otros modelos como puede ser uno basado en un árbol de decisión o simplemente suponer que todas las mujeres sobrevivieron y que todos los hombres perecieron. La realización de estos dos tutoriales se adjunta en esta memoria a modo de apéndices.\\

\section{Exploración de datos} \emph{Ver archivo \archive{exploracion.Rmd}}\\

En primer lugar vamos a presentar el conjunto de datos que tenemos, disponemos de 891 instancias en el conjunto de entrenamiento y 418 en el conjunto de test. Estas instancias presentan el siguiente conjunto de atributos:

\begin{enumerate}
\item \textbf{PassengerId: } identificador del pasajero.
\item \textbf{Pclass: } la clase en la que embarcó.
\item \textbf{Name:} el nombre del pasajero.
\item \textbf{Sex: } sexo del pasajero.
\item \textbf{Age: } edad del pasajero.
\item \textbf{SibSp: } número de hermanos/esposos/esposas del pasajero que viajaban también a bordo.
\item \textbf{Parch: } número de padres/hijos del pasajero que viajaban también a bordo.
\item \textbf{Ticket: } número o identificador del ticket de embarque del pasajero.
\item \textbf{Fare: } lo que pagó el pasajero por su pasaje.
\item \textbf{Cabin: } camarote(s) en el(los) que viajó el pasajero.
\item \textbf{Embarked: } puerto desde el que embarcó. C = Cherbourg, Q = Queenstown, S = Southampton.
\item \textbf{Survived: } el pasajero murió (0) o sobrevivió (1). La variable a predecir y que por tanto no está presente en las intancias del conjunto de test.
\end{enumerate}

Lo primero que hemos hecho ha sido estudiar si contábamos con valores perdidos. Gracias a los tutoriales notamos que hay algunos atributos que, si bien no presentan valores perdido (\textbf{NA}), contienen una cadena vacía, lo que se podría considerar también un valor perdido, y por lo tanto esto también se tendrá en cuenta en la fase de preprocesamiento. Tras este análisis vemos que la mayoría de valores perdidos están en el atributo \textbf{Age}, también hay algún valor perdido en el \textbf{Embarked} y muchos valores perdidos para el atributo \textbf{Cabin} (señalar aquí que en R es mejor emplear los operandos condiciones \code{\&} y \code{|}, en lugar de \texttt{\&\&} o \texttt{||}, ya que con estos obtenemos resultados incorrectos). Para estudiar de forma general el conjunto de datos de los que disponemos han sido de utilidad los métodos \code{summary} y \code{str}.\\

De hecho en ambos conjuntos, para el atributo \textbf{Cabin} tenemos aproximadamente un 78\% de valores perdidos, lo cual es una cantidad muy elevada. En el tutorial sobre ingeniería de características que se facilita desde la propia página de la competición se habla de obtener, procesando este campo, la cubierta en la que se alojó el pasajero. Sin embargo, puesto que se trata de un atributo con tantos valores perdidos se ha decidido no realizar dicho preprocesamiento: no tenemos información suficiente para decidir en qué cubierta viajó el pasajero y una imputación, usando por ejemplo el paquete \code{mice}, no sería de mucha confianza al tener un número tan elevado de valores perdidos.\\

Vamos a ver ahora la primera gráfica sobre nuestro conjunto de datos, lo que vamos a reflejar en esta gráfica es el desequilibrio entre las dos clases (549 muertos y 342 sobrevivientes), aunque no se trata de un desequilibrio demasiado grande lo trataremos en la fase de preprocesamiento y trataremos de comparar distintas técnicas de balanceo de clases según su rendimiento para un algoritmo determinado:

\begin{figure}[H]
  \centering
  \includegraphics[width=\textwidth]{imgs/imbalanced.pdf}
  \caption{Desbalanceo entre clases}
\end{figure}

En la gráfica anterior también mostramos en qué proporción sobreviven hombres y mujeres en el conjunto de entrenamiento. Como podemos ver son las mujeres las que en su mayoría sobrevivieron mientras que los hombres tuvieron menos posibilidades, recordemos que durante el accidente del Titanic se estableció la política de evacuar a mujeres y niños en primer lugar. Este hecho nos lleva, en el tutorial de Trevor Stephens, a plantear el modelo sexista.

\begin{figure}[H]
  \centering
  \includegraphics[width=\textwidth]{imgs/classes.pdf}
  \caption{Distribución de sobrevivientes según la clase}
\end{figure}

En esta gráfica lo que vemos es cómo se distribuyen los sobrevivientes en las distintas clases de los pasajeros. Como vemos son los de tercera clase los que perecieron en mayor proporción. Sin embargo observamos algo interesante, y es que en los sobrevivientes lo parece haber tanta diferencia entre las clases, siendo los pasajeros de segunda clase y no los de tercera los que murieron en mayor proporción. Como vemos el análisis exploratorio ha sido principalemente enfocado a ver cómo influyen las distintas características de los pasajeros en su probabilidad de sobrevivir, al fin y al cabo este es el problema que se nos plantea con este dataset.\\

\begin{figure}[H]
  \centering
  \includegraphics[width=\textwidth]{imgs/age.pdf}
  \caption{Distribución de sobrevivientes según la edad y el sexo.}
\end{figure}

Aquí lo que podemos apreciar es en primer lugar que las mujeres sobreviven en mayor proporción que los hombres, por otro lado los hombres jóvenes parecen sobrevivir en mayor proporción que los mayores, al contrario que ocurre con las mujeres.

\begin{figure}[H]
  \centering
  \includegraphics[width=\textwidth]{imgs/fare.pdf}
  \caption{Distribución de sobrevivientes según la edad y el sexo.}
\end{figure}

En esta gráfica apreciamos algo parecido a lo que teníamos en la gráfica con las clases de los pasajeros. Los pasajeros que pagaron menos por su pasaje, los de menor clase suponemos, son los que mueren en mayor proporción. Claro hay muerte que no se corresponden con esto, pensemos en que aquí estamos analizando los factores uno por uno por separado. Hay muchos factores distintos que pudieron influir en la muerte de un pasajero, entre otros su localización en el barco. Como ya hemos dicho este no es un factor que nosotros hayamos tenido en cuenta, y queda como trabajo futuro.\\

Por último vamos a ver cómo se distribuye la tasa de muertos según el tamaño de la familia. Parece lógico pensar que se intentó evacuar a las familias juntas, o que estas presionaron para que se evacuasen a todos sus miembros. Sin embargo, familias con un tamaño muy grande es más complicado que se pudiesen evacuar y quizás, por tal de permanecer unidas perecieran. Por tanto esta variable que se crea nueva, el tamaño de la familia, puede resultar muy interesante:

\begin{figure}[H]
  \centering
  \includegraphics[width=\textwidth]{imgs/fsize.pdf}
  \caption{Distribución de sobrevivientes según el tamaño de su familia.}
\end{figure}

Como podemos ver aquellos pasajeros que viajaban solos tenían mayor probabilidad de morir y por otro lado, las familias con un tamaño demasiado grande, cinco miembros o más, también lo tenían más difícil para sobrevivir.



\section{Exploración de datos}

En primer lugar, tras descargar los distintos conjuntos de imágenes pasamos a analizar simplemente el balanceo entre las clases. Nos encontramos antes un problema de clasificación de imágenes y en un campo como es el de la oncología en el que no tenemos experiencia ninguna, por tanto no podemos hacer un análisis mucho más extenso que éste. Al fin y al cabo lo que buscamos entrenando distintas redes neuronales es que sean estas, con su capacidad de aprendizaje, las que nos ayuden a determinar y extraer las características más relevantes de las imágenes para la clasificación que nos ocupa. En la siguiente se muestra un histograma con el número de imáges de cada clase, además se ha diferenciado entre las imágenes del conjunto base y las imágenes adicionales que se proporcionan en la misma plataforma para contar con más ejemplos que nos ayuden a entrenar nuestras redes:\\

1644 4066 2775\\

Como podemos ver las clases presentar un cierto desbalanceo, siendo el cáncer tipo 2 del que más muestras disponemos. Con lo cual estaríamos trabajando en un problema desbalanceado, no obstante en esta práctica no se a tratado este problema por falta de tiempo, quedando el balanceo de las clases como trabajo futuro.    


\section{Preprocesamiento}

En un primer momento se trabajó con las imágenes originales descargas desde Kaggle. Pero rápidamente, nos dimos cuenta de que debido a su tamaño, no iban a ser tratables mediante las herramientas disponibles. Con lo cual, el profesor aconsejó y facilitó las imágenes con un escalado, cuyo tamaño es de \emp{256x256 píxeles}. Con estas imágenes, no hubo ningún tipo de problema al utilizar las herramientas y técnicas que se comentarán a continuación.\\

Además, nuestro compañero Francisco Javier Bolivar Lupiáñez, nos facilitó las imágenes adicionales reescaladas utilizando el mismo procedimiento que el profesor. Esto nos fue de mucha ayuda ya que por un lado, descargar las imágenes desde Kaggle tomaba mucho tiempo y además, se producían fallos en la descarga. Por otro lado, se ahorró el tiempo de cómputo necesario para redimensionar las imágenes, dicho tiempo podría haber sido muy abultado ya que estas imágenes se cuentan por miles.\\

También se realizaron otros escalados, \emp{de 32x32 y 244x244}. El primero fue decisión propia, para realizar unas primeras pruebas con poco tiempo de cómputo, aunque, como veremos mas adelante, se obtuvieron en algunos casos mejores resultados que con las imágenes de 256x256. El segundo, era para poder adaptar nuestras imágenes a la entrada esperada por las redes neuronales que utilizaron en la fase de \textit{fine tuning}. Esto último se verá con mas detalle en las secciones posteriores.\\

En el script que se uso como base, se tuvo que eliminar una transposición de las imágenes, ya que alteraba el orden de las dimensiones de las matrices que representaban las imágenes, lo que producía errores en el proceso de \textit{fine tuning}.\\

Se optó por realizar un \emp{\textit{data augmentation} sobre las imágenes originales}, dicho aumento se realizó de manera online, es decir, en cada fase del entrenamiento de la red neuronal se creaba un nuevo \textit{batch} de imágenes. Para ello se emplearon una serie de transformaciones como: rotaciones, escalados, zoom, movimiento de cizalla aleatorios. Esto se hizo con la herramienta \code{ImageDataGenerator} de la librería \code{Keras}. También se realizó un proceso similar pero más específico en el caso del uso del descript HOG, hablaremos de esta idea más adelante.\\

Finalmente, se pasó las imágenes a \emp{escala de grises} para realizar extracción de características, tanto con \textbf{HOG}, como con \textbf{BRIEF}. Aunque, finalmente, no se pudo obtener resultados debido al poco tiempo disponible para realizar esta última fase. Con esto terminaríamos el repaso a las distintas transformaciones que se han realizado sobre las imágenes antes de pasar a entrenar una red o realizar la extracción de características.


\section{Técnicas de clasificación}

\subsection{OVO}

Tras los experimentos realizados anteriormente pasamos a descomponer nuestro problema multiclase con un esquema OVO para ello entrenamos tres redes neuronales distintas, la elección de la red a entrenar la hicimos en base a los experimentos anteriores, con lo cual para cada uno de los tres problemas binarios entrenamos una ResNet 50 modificando sus capas \textit{fully connected} para que tengamos dos salidas en lugar de tres como anteriormente.\\

Se generaron, para evitar problemas con las etiquetas de cada clase, de modo que se estuviesen siempre en el conjunto {0,1} y ahora simplemente entrenamos cada una de las redes sobre cada uno de los conjunto creados, nuevamente con 10 \textit{epochs} de \textit{transfer learning} y otras 50 de \textit{fine tunning}. Una vez se han entrenado cada una de las redes pasamos al esquema de agregación de los resultados, las probabilidades, obtenidas para cada una de las clases. Señalar que para estos esquemas se ha supuesto que cada uno de las redes, para aquella clase que ignoran que existe, dan como probabilidad el valor 0.\\

El primer esquema que se nos ocurrió fue el más simple de todos, simplemente para cada clase la probabilidad asignada será \emp{la media} de la probabilidad para dicha clase asignada por cada una de las tres redes neuronales. No obstante este esquema tiene algo que no nos gusta y es que si por ejemplo tenemos las siguientes probabilidades:

\begin{table}[H]
\centering
\caption{Ejemplo de probabilidades}
\label{my-label}
\begin{tabular}{|c|c|c|c|}
\hline
Clasificador \textbackslash Clase & Tipo 1 & Tipo 2 & Tipo 3 \\ \hline
1-2                               & 1      & 0      & 0      \\ \hline
1-3                               & 1      & 0      & 0      \\ \hline
2-3                               & 0      & 0.5    & 0.5    \\ \hline
\end{tabular}
\end{table}

Entonces a esta imagen, usando como esquema de agregación la media, tenemos las siguientes probabilidades:

\begin{table}[H]
\centering
\caption{Probabilidades de la media}
\label{my-label}
\begin{tabular}{|c|c|c|}
\hline
Tipo 1 & Tipo 2 & Tipo 3 \\ \hline
0.67   & 0.19   & 0.14   \\ \hline
\end{tabular}
\end{table}

Como vemos mientras que dos clasificadores nos dan una confianza del 100\% en que la imagen es del Tipo 1, como el clasificador 2-3 da como probabilidad para esa clase el 0, puesto que no conoce el Tipo 1, y aunque ese clasificador no tiene certidumbre sobre la pertenencia de la imagen al Tipo 2 o al Tipo 3, dando probabilidad 0.5 para ambas, lo que ocurre es que la probabilidad que le damos a la imagen para el Tipo 1 es de 0.67. Esto nos parecía algo injusto y nos gustaría un esquema que tuviese en cuenta este tipo de escenarios, dando mayor peso a los dos primeros clasificadores que al que no está seguro sobre la clase a la que pertenece la imagen.\\

Por lo que acabamos de exponer decidimos probar con otro esquema, para ello leímo un artículo, en el que participa como autor el profesor Francisco Herrera y que referenciamos en la bibliografía de esta práctica, en el que se describen distintos esquemas de agregación para binarización de problemas OVO y OVA. Entre todos los esquemas expuestos nos decantamos por el llamado \emp{LVPC}, el cual se dice en el artículo que, en el caso en estar ante un problema con clasificadores binarios normalizados, es decir, clasificadores que para una clase den la probabilidad $\alpha$ y para la otra clase den su opuesto, $1 - \alpha$, entonces estamos antes un esquema establece una votación ponderada que penaliza a los clasificadores que no tienen confianza en su elección respecto a la pertenencia de la instancia a una de las dos clases que distingue.\\

Este esquema funciona como sigue, si llamamos $r_{ij}$ a la probabilidad que da el clasificador binario para las clases $i$ y $j$ para la pertenencia de una instancia a la clase $i$, y $r_{ji}$ a su análogo para la clase $j$. Entonces se calculan los siguientes términos:

\begin{center}
$P_{ij} = r_{ij} - min\lbrace r_{ij}, r_{ji} \rbrace$\\
$P_{ji} = r_{ji} - min \lbrace r_{ij}, r_{ji} \rbrace$\\
$C_{ij} = min\lbrace r_{ij}, r_{ji} \rbrace$\\
$I_{ij} = 1 - max\lbrace r_{ij}, r_{ji} \rbrace$
\end{center}

donde, como explica el artículo $C_{ij}$ es el grado de conflicto (el grado en el que ambas clases son escogidas por el clasificador), $I_{ij}$ es el grado de ignorancia que tiene el clasificador (el grado en el que ninguna de las dos clases es escogida por el clasificador) y finalmente $P_{ij}$ y $P_{ji}$ es el grado de preferencia del clasificador hacia cada una de las dos clases. Entonces una vez que hemos calculado esto para cada instancia el esquema asigna la siguiente clase a esta instancia:

\begin{center}
$\underset{i = 1,...,m}{argmax} \underset{1 \leq j \neq i \leq m}{\sum} P_{ij} + \frac{1}{2} C_{ij} + \frac{N_i}{N_i + N_j}I_{ij}$
\end{center}

Donde $N_i$ es el número de instancias en nuestro conjunto de entrenamiento de la clase $i$. En nuestro caso, como realizamos un \textit{data augmentation online} entonces hemos supuesto que se mantienen las proporciones del conjunto de entrenamiento original, las del conjunto inicial y las adicionales para cada clase. Por otro lado como no queremos dar un resultado absoluto, es decir, dar probabilidad 1 a una clase y 0 al resto, lo que hemos hecho es en lugar de tomar el $argmax$ de las 3 sumatorias que se calculan es usar directamente el resultado de las sumatorias, dividido por 3, ya que nos encontramos con que el valor de las tres sumatorias sumaban 3 siempre, con lo que hemos dividido por tres para que sumen 1.

\subsection{Extracción de características con una CNN}



\subsection{Extracción de características con técnicas convencionales}

\subsubsection{HOG}

\subsubsection{ORB}

\section{Presentación y discusión de resultados}

Los primeros experimentos que se realizaron fue empleando la técnica de \emp{\textit{learning from scratch}}, es decir, diseñar y entrenar una red desde cero. Para ello se empleó la red que se definía en un script de Kaggle en un \textit{kernel} de esta competición y que se enlaza en la bibliografía. Aunque debido a nuestra inexperiencia en el ámbito del \textit{deep learning} no supimos cómo mejorar nuestra red o hacer su entrenamiento más efectivo de cara a la clasificación, si que pudimos observar algunos resultados bastante interesantes y que se explican a continuación.\\

En esta fase hicimos pruebas tanto con las imágenes rescaladas a 256x256 como a 32x32, este último rescalado fue algo que se hizo para simplemente poder realizar unas primeras pruebas sin demasiado tiempo de cómputo, no obstante resultaron ser unos experimentos más valiosos de lo que esperábamos. Y es que como podemos ver en \autoref{soluciones} se obtuvieron mejores resultados trabajando con imágenes de 32x32 que con las de mayor tamaño.\\

Realmente nosotros esperábamos que los resultados fuesen peores, de hecho una de nuestras preocupaciones de cara a afrontar la práctica era tener que reescalar las imágenes a un tamaño más reducido si los tiempos de cómputo eran demasiado elevados, esta preocupación se debía a que pensamos que cuanto más pequeña hiciésemos la imagen más información estaríamos perdiendo. Realmente lo que creemos que está influyendo en estos resultados en la simplicidad de la red que estamos estuadiando.\\

Lo que puede estar ocurriendo es que al tener una red tan simple ésta no sea capaz de extraer todos los matices que puede haber en una imagen y que por tanto la red se esté \textit{saturando de información} y por consiguiente esté sobreaprendiendo, por esto estamos obteniendo estos resultados. De hecho esto se observa bien cuando intentamos usar además del conjunto de entrenamiento base las imágenes adiconales que se facilitan en Kaggle. Mientras que en el caso de trabajar con imágenes pequeñas se mejora el score obtenido para las imágenes de 256x256 la red se \textit{satura} aún más y el score empeora. Observemos que el \textit{accuracy} en el conjunto de validación (un 40\% del conjunto de entrenamiento) no parece dar mucha información, en la mayoría de casos es similar, y cuando parece que nuestra red está sobreaprendiendo, en la \textit{submission} 3, el resultado obtenido es mejor que en el caso de las imágenes de 256x256 con un \textit{val\_acc} menor.\\

Distinto es cuando pasamos a hacer \emp{\textit{data augmentation}} siendo ahora la red entrenada con las imágens más grande la que mejores resultados obtiene, esto puede deberse a que al proporcionarla a la red, en cada \textit{epoch}, nuevos ejemplos conseguimos que el sobreaprendizaje que antes estábamos sufriendo se suavice pues ahora tiene más ejemplos sobre los que extraer características y por tanto puede realizar un aprendizaje más diversificado con su limitada capacidad por tratarse de un red relativamente simple. Si que vemos que cuando aumentamos el número de nuevas imágenes se generan, en la \textit{submission} 11, el resultado empeora ligéramente, quizás ya estemos nuevamente saturando la red al dar demasiados ejemplos que, por mucho que intentemos modificarlos a través de transformaciones aleatorias, no dejan de ser imágenes del conjunto de train original.\\

Cuando pasamos al proceso de \emp{\textit{fine tuning}} observamos que mientras que Inception V3 no mejoraba los resultados obtenidos hasta el momento, aunque sí que se obtenía una puntuación similar a los mejores resultados obtenidos, fue ResNet50 la red que mejoró nuestro resultados significativamente. Esto no quiere decir nada, de hecho no tenemos el conocimiento necesario como para poder analizar por qué con los mismo parámetros de entrenamiento una red da mejores resultados que otra. Lo que sí creemos es que al ser Inception una red de mayor tamaño (o al menos así nos lo ha parecido al obtener el número de capas de las redes implementadas en \code{Keras}) quizás necesite de más \textit{epochs} para poder aprender correctamente o que por otro lado al ser una red tan grande lo que estemos sufriendo sea de sobreaprendizaje, serían necesarios más experimentos para poder deducirlo.\\

Por otro lado, y dado que con ResNet50 obtuvimos nuestros mejores resultados hasta la fecha intentamos mejorarlos más entrenando la red durante más \textit{epochs}, para ello cargamos la última red que habíamos entrenado (gracias a guardar los pesos resultado del entrenamiento previo) y a partir de ahí entrenar la red durante 100 épocas más. Sin embargo los resultados fueron desastrosos producto, seguramente, del sobreaprendizaje. Por tanto aquí se nos plantea otra cuestión, ¿habría sido mejor usar como red para la predicción un de las etapas previas del entrenamiento de ResNet50 en lugar de usar la que mejor resultado de \textit{val\_acc} obtuvo?\\

Pasamos ahora a hablar de la técnica con la mejor resultados obtuvimos, \emp{la binarización del problema por medio de un esquema OVO}. Para ello, y a la luz de los resultados anteriores entrenamos (mediante nuevamente un proceso de \textit{fine tuning}) 3 ResNet50 que distinguiesen entre dos de las tres clases de nuestro problema. Una vez entrenadas y obtenidas las predicciones de cada una de ellas las combinamos mediante dos esquemas de agregación. Los mejores resultados fueron los obtenidos usando la media frente al esquema LVPC. La única razón que se nos ocurre de este resultado es que dado que el esquema LVPC penaliza a los clasificadores que sufren de incertidumbre en su decisión quizás lo que estemos haciendo sea \textit{obviar} una indecisión justificada y por tanto estemos equivocándonos al dar mayor peso a una clase que realmente no es la correcta.\\

Pasamos finalmente a los experimentos en los que aplicamos esquemas de aprendizaje automático como \emp{SVM o GBoost sobre características extraídas}. Desgraciadamente por cuestiones de tiempo que ya se han comentado anteriormente no se ha podido hacer una comparativa usando características extraídas con técnicas usuales del campo de la Visión por Computador, en nuestro caso hemos trabajado con características extraídas con la red neuronal (que distinguía entre las tres clases del problema), la ResNet50 reentrenada. Así hemos usado dos esquemas SVM y GBoost, siendo el primero el que mejores resultados de clasificación nos ha dado. Y observamos además que obtenemos peores resultados empleando los mejores parámetros dentro del \textit{grid} definido (que incluía los que se usaron en un primer intento) para GBoost que usando unos parámetros por defecto o al azar. Teóricamente Gradiente Boosting es una técnica que no es muy sensible al sobreaprendizaje (además que en el caso de SVM elegimos el parámetro que hace que nos ajustemos a los datos de entrenamiento lo máximo posible con el \textit{grid} de parámetros que definimos), sin embargo parece que al elegir el número más elevado de árboles de nuestro \textit{grid} y con la mayor profundidad posible esto ha podido hacer que nuestro modelo sobreaprenda. Además parece que las máquinas de soporte vectorial se han adaptado mejor a nuestro espacio de más de 2000 características que los árbol del gradient boosting.\\

Finalmente señalar que se han obtenido mejores resultados empleando CNNs directamente que realizando la extracción de características, si bien es cierto que estos resultados no son comparables puesto que no se ha realizado \textit{data augmentation} al extraer las características.







\section{Conclusiones y trabajo futuro}
TODO

\begin{landscape}
\section{Listado de soluciones}
\pagestyle{empty}
% Please add the following required packages to your document preamble:
% \usepackage[table,xcdraw]{xcolor}
% If you use beamer only pass "xcolor=table" option, i.e. \documentclass[xcolor=table]{beamer}
\begin{table}[H]
\centering
\caption{My caption}
\label{my-label}
\begin{tabular}{|l|c|c|c|c|c|}
\hline
\rowcolor[HTML]{9B9B9B} 
\multicolumn{1}{|c|}{\cellcolor[HTML]{9B9B9B}\textbf{Nº}} & \textbf{CSV}                                                 & \textbf{Preprocesamiento} & \textbf{Algoritmos y softw.} & \textbf{val\_acc}                                                   & \textbf{Kaggle score} \\ \hline
1 6/6                                                     & submission 0001.csv                                          &                           &                              & desc.                                                               & 0.92302               \\ \hline
2 7/6                                                     & scrath\_train.csv                                            &                           &                              & 0.55818                                                             & 0.92872               \\ \hline
3 7/6                                                     & scrath\_train32.csv                                          &                           &                              & 0.85533                                                             & 0.89738               \\ \hline
4 7/6                                                     & scrath\_train256simplerCNN.csv                               &                           &                              & 0.54469                                                             & 0.94776               \\ \hline
\rowcolor[HTML]{FD6864} 
5 7/6                                                     & scrath\_train\_all32.csv                                     &                           &                              & desc.                                                               & 2.31209               \\ \hline
6 7/6                                                     & scrath\_train\_all32v2.csv                                   &                           &                              & 0.51296                                                             & 0.91077               \\ \hline
7 8/6                                                     & scrath\_train\_all256.csv                                    &                           &                              & 0.56040                                                             & 1.69263               \\ \hline
\rowcolor[HTML]{FD6864} 
8 8/6                                                     & scrath\_train\_all32dataAugmentation.csv                     &                           &                              &                                                                     & 0.90680               \\ \hline
9 8/6                                                     & scrath\_train\_all32dataAugmentation.csv                     &                           &                              & 0.49087                                                             & 0.90646               \\ \hline
10 8/6                                                    & scrath\_train\_all256dataAugmentation.csv                    &                           &                              & 0.52622                                                             & 0.87450               \\ \hline
11 9/6                                                    & scrath\_train\_all256dataAugmentationx20.csv                 &                           &                              & 0.52298                                                             & 0.88144               \\ \hline
\rowcolor[HTML]{FD6864} 
12 10/6                                                   & fine\_tunning\_InceptionV3\_all244fineTuningx20.csv          &                           &                              & desc.                                                               & 0.95942               \\ \hline
13 10/6                                                   & fine\_tunning\_InceptionV3\_all244fineTuningx20.csv          &                           &                              & 0.62110                                                             & 0.87606               \\ \hline
14 11/6                                                   & fine\_tunning\_ResNet50\_all244fineTuningx20.csv             &                           &                              & 0.59959                                                             & 0.82786               \\ \hline
15 12/6                                                   & fine\_tunning\_ResNet50\_all244fineTuningx20Continuation.csv &                           &                              & 0.65468                                                             & 1.00893               \\ \hline
\rowcolor[HTML]{67FD9A} 
16 12/6                                                   & fine\_tunning\_ResNet50\_all244fineTuningx20OVOMedia.csv     &                           &                              & \begin{tabular}[c]{@{}c@{}}0.75701\\ 0.81900\\ 0.67885\end{tabular} & 0.81792               \\ \hline
\rowcolor[HTML]{FD6864} 
17 12/6                                                   & fine\_tunning\_ResNet50\_all244fineTuningx20OVOOtro.csv      &                           &                              & desc.                                                               & 0.81792               \\ \hline
18 12/6                                                   & fine\_tunning\_ResNet50\_all244fineTuningx20OVOLVPC.csv      &                           &                              & \begin{tabular}[c]{@{}c@{}}0.75701\\ 0.81900\\ 0.67885\end{tabular} & 0.82102               \\ \hline
19 12/6                                                   & ResNet\_SVM.csv                                              &                           &                              & 0.59959                                                             & 0.91983               \\ \hline
20 13/6                                                   & SVMprobsTune.csv                                             &                           &                              & 0.59959                                                             & 0.87503               \\ \hline
21 13/6                                                   & GBoostprobsTune.csv                                          &                           &                              & 0.59959                                                             & 0.94694               \\ \hline
22 13/6                                                   & ResNet\_GBoost.csv                                           &                           &                              & 0.59959                                                             & 0.90360               \\ \hline
\end{tabular}
\end{table}
\end{landscape}

\newpage
hola

\section{Bibliografía}
\setcounter{page}{17}
\begin{itemize}
\item \href{https://arxiv.org/pdf/1512.03385.pdf}{\textit{Paper} donde se define ResNet}
\item \href{https://arxiv.org/pdf/1409.4842.pdf}{\textit{Paper} donde se define InceptionV3}
\end{itemize}



\end{document}
