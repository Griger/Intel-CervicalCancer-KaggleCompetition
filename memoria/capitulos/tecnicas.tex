\section{Técnicas de clasificación}

\subsection{Learning from scratch}

Como primera técnica vamos a utilizar \textit{learning from scratch}. La idea básica de esta técnica es crear una red neuronal, seleccionando manualmente las capas, es decir, estableciendo tu topología, y posteriormente entrenarla. En un primer momento decidimos crear una red neuronal con una serie de capas de convolución, maxpolling, dropout, flatten y dense. Para ello creamos un modelo utilizando la clase \code{Sequencial()} de \code{Keras}. Una vez creado el modelo, se le añaden las distintas capas usando el método \code{add}, al que le pasamos el tipo de capa y sus parámetros. Estos parámetros varían dependiendo del tipo de capa.

\begin{itemize}
\item Para la capa de Convolución disponemos de los parámetros \textit{input\_shape}, que es el tamaño de la entrada, \textit{activation}, función de activación, el número de convoluciones a realizar y el tamaño de la matriz, esto viene de la forma (4,3,3), 4 convoluciones con una matriz 3x3 \textit{dim\_ordering}, y el orden de la dimensión de la imagen. Este último es compartido por el resto de capas. 
\item Para la capa de MaxPolling disponemos de \textit{pool\_size}, tamaño de la ventana de polling, y \textit{strides}, factor por el cual reducir la escala
\item Para la capa de Dropout contamos con \textit{rate}, cantidad de información a desechar.
\item Para la capa de Dense necesitamos \textit{activation}, función de activación, y las dimensiones de la salida.
\item La capa Flatten no requiere ningún parámetro.
\end{itemize}

Finalmente el modelo con las distintas capas se compila usando \code{compile()} que recibe como parámetros el optimizador a usar y la función de pérdida.\\

En nuestro caso, la primera red con la que se realizaron pruebas contaba con la siguiente topología:

\begin{itemize}
\item Convolución aplicando 4 convoluciones con una matriz 3x3, con activación relu e input\_shape (3,16,16).
\item MaxPooling2D con pool\_size (2, 2) y strides (2, 2).
\item Convolución aplicando 8 convoluciones con una matriz 3x3, con activación relu e input\_shape (3,16,16).
\item MaxPooling2D con pool\_size (2, 2) y strides (2, 2).
\item Dropout con rate 0.2
\item Flatten
\item Dense con dimesión 12 y activation tanh
\item Dropout con rate 0.1
\item Dense con dimesión 3 y activation softmax
\end{itemize}

Con dicha red se realizaron los primeros experimentos, con \texti{input\_shapes} de 32x32 y 256x256. Al realizarlos se obtuvieron mejores resultados con las imágenes de menor tamaño, por lo que se pensó que tal vez la red sobreaprendia con imágenes grandes. Al llegar a esta conclusión se eliminaron una serie de capas del modelo para corroborar que nuestras deducciones eran correctas. Pero el resultado fue una puntuación peor que la obtenida con la topología inicial. Debido a esto, este modelo mas sencillo fue desechado para posteriores expermientos.\\

Llegados a este punto, decidimos aumentar el número de capas de nuestra red, por lo que su topología varió de la siguiente manera:

\begin{itemize}
\item Convolución aplicando 1 convolucón con una matriz 3x3 con activación relu e input\_shape (3, 256, 256)
\item MaxPooling2D con pool\_size (2, 2) y strides (2, 2)
\item Convolución aplicando 2 convoluciones con una matriz 3x3 y activation relu
\item MaxPooling2D con pool\_size (2, 2) y strides=(2, 2)
\item Convolución aplicando 4 convoluciones con una matriz 3x3 y activation relu
\item MaxPooling2D con pool\_size (2, 2) y strides (2, 2)
\item Convolución, aplicando 8 convoluciones con una matriz 3x3 y activación relu
\item MaxPooling2D con pool\_size (2, 2) y strides (2, 2).
\item Dropout con rate 0.2
\item Flatten
\item Dense con dimensión 24 y activación tanh
\item Dropout con rate 0.1
\item Dense con dimensión 12 y activación tanh
\item Dropout con rate 0.1
\item Dense con dimesión 3 y activación softmax
\end{itemize}

Con este modelo fue con el que mejores resultados se obtuvieron. En un primer momento se comprobó que este nuevo modelo mejoraba los resultados del anterior, una vez verificado procedimos a entrenarlo, pero en este caso utilizando las imágenes extra junto a las imágenes base. Esto no se realizó anteriormente ya que queríamos obtener una buena red ya que no sabíamos el tiempo de cómputo que requeriría entrenar utilizando todas las imágenes. Se realizaron experimentos con un tamaño de entrada de imágenes de 32x32 y de 256x256. Como era de esperar, los resultados mejoraron a los anteriores (PON AQUI SI 32 ERA MEJOR QUE 256 O AL REVES). Todos estas pruebas se realizaron con un
\textit{batch\_size} de 15, y entre 50 y 100 épocas. Viendo esta tendencia a la mejora, decidimos dar le salto a \emp{data augmentation}.\\

Para realizar el \textit{data augmentation} se utilizó el método \code{ImageDataGenerator} del paquete \code{preprocessing.image}. Dicho método tiene una gran cantidad de parámetros que nos sirven para crear nuevas imágenes a partir de las disponibles de diversas maneras, en nuestro caso hemos utilizado los siguientes:

\begin{itemize}
\item rotation\_range con valor 180. Básicamente establece el rango en el cual se van a realizar rotaciones aleatorias.
\item width\_shift\_range y height\_shift\_range, ambos con valor 0.2. Rangos para cambios aleatorios horizontes y verticales.
\item shear\_range con valor 0.2. Para realizar transformaciones de tipo cizalla.
\item zoom\_range con valor 0.2. Para realizar transformaciones de tipo zoom.
\item horizontal\_flip y vertical\_flip con valor True. Para invertir las imágenes vertical y horizontalmente.
\item fill\_mode con valor nearest. Los puntos fuera de los límites de la entrada son rellenados dependiendo del valor elegido.
\end{itemize}

Una vez obtenidas las nuevas imágenes, son agregadas las imágenes de entrenamiento usando \code{fit}. También debemos modificar el valor del parámetro de \textit{steps\_per\_epoch} por el tamaño del conjunto de entrenamiento/15 (valor de \textit{batch\_size}) y multiplicarlo por un otro valor, esto se va a comentar a continuación. Con este valor podemos aumentar el número de pasos que se realizan antes de avanzar a la siguiente época, lo que significa que si aumentamos el valor, se aumenta la cantidad de datos adicionales que se procesan. Dicho esto, se realizaron pruebas con tamaños de entrada 32x32 y 256x256. Para 32x32 el valor multiplicador comentado fue de 4, mientras que para 256x256 fue 10 y posteriormente 20, ya que se obtuvo un mejor resultado usando las imágenes de tamaño 256x256 que las de 32x32, al contrario de lo que pasó anteriormente. Como comentario, utilizando \textit{data augmentation} x20 se obtuvo uno de los mejores resultados.

\subsection{Fine tunning}

A continuación vamos a hablar sobre \textit{fine tunning} y los experimentos ejecutados utilizando esta técnica. En esta ocasión se cargaron 2 redes sobre las cuales se realizó \textit{fine tunning}. Dichas redes son, \emp{ResNet50} y \emp{InceptionV3}. Si queremos realizar un entrenamiento de una manera estable y consistente debemos realizar un procedimiento denonimado \textit{transfer learning} antes de aplicar el propio \textit{fine tunning}. \textit{Transfer learning} consiste básicamente en congelar todas las capas excepto la penúltima y volver a entrenar la última capa densa. Por otro lado, \textit{fine tunning} consiste en descongelar las capas convolucionales inferiores y reentrenar mas capas.\\

A diferencia de \textit{Learning from scratch} se han realizado un menor número de pruebas, ya que el tiempo de ejecución es mayor y habia que emplearlo correctamente si se querian comprobar todas las cosas que se han comentando en esta memoria. Por lo que hemos realizado dos experimentos con InceptionV3 y tres para ResNet50, ya que como se verá a continuación, ofrece mejores resultados. Cabe mencionar que ambas redes esperan las entradas en tamaño 244x244, siendo la primera vez que se trabajó con dicho tamaño, y a su vez, se usan todas las imágenes disponibles, base y extra, y \textit{data augmentation} en los experimentos. Sin mas dilación, vamos a hablar con mas profundida de ambas redes.\\
 
\subsubsection{InceptionV3}

Este modelo cuenta con una serie de pesos preentrenados usando ImageNet. Lo primero que debemos hacer es crear el modelo usando \code{InceptionV3} del paquete \code{keras.applications.inception\_v3}, a dicho modelo le añadimos una capa de agrupación promedio espacial global, \textit{global spatial average pooling layer}, con \code{GlobalAveragePooling2D} del paquete \code{keras.layers}. También le agregamos una capa densa, de dimensión 3 y función de activación softmax. Una vez que disponemos del modelo configurado correctamente, se procede a entrenarlo. Para ello, como hemos mencionado antes, se realiza, en primer lugar \textit{transfer learning} con un total de 10 épocas, el resto de parámetros se mantienen como en \textit{Learning from scratch}, ya que fueron los que mejores resultados proporcionaron. Una vez acabado, se procede a con \textit{fine tunning}. En este caso, con un total de 50 épocas. En esta red congelamos las 174 primeras capas. Para descongelarlas usamos el método \code{compile} de la clase \code{model} con \textit{optimizer} SGD, \textit{learning rate} 0.0001, \textit{momentum} 0.9, y como función de pérdida/loss, \textit{sparse\_categorical\_crossentropy}. Se obtuvieron buenos resultados, aunque serían superados por los proporcionados por \emp{RestNet50}

\subsubsection{ResNet50}

Este modelo, en cuanto a funcionamiento por parte del usuario, es muy similar al modelo anterior, por lo que realmente, se podría utilizar un script de en el cual se ha entrenado a InceptionV3 para hacer lo propio con ResNet50, realizando los cambios adecuados, como es natural. Para poder usar \code{ResNet50} necesitamos el paquete \code{keras.applications.resnet50}. Al modelo, se le añaden las mismas capas que se le añadieron a InceptionV3, pero en este caso, se congelan las 162 primeras capas, ya que esta red cuenta con un menor número de ellas y se dividen en lo que podríamos denonimar "patrones", por lo que decidimos congelar todas las capas hasta el inicio del último de estos "patrones". Las épocas se mantienen iguales, 10 para \textit{transfer learning} y 50 para \textit{fine tunning}. Utilizando esta red se obtuvieron resultados muy prometedores, los mejores obtenidos, por lo que se decidió a guardar el modelo entrenado, y volver a entrenarlo usando únicamente \textit{fine tunning} con 100 épocas, lo que nos daría un total de 150 épocas durante su entrenamiento. Para nuestra decepción, los resultados no mejoraron.































 







\subsection{OVO}

Tras los experimentos realizados anteriormente pasamos a descomponer nuestro problema multiclase con un esquema OVO para ello entrenamos tres redes neuronales distintas, la elección de la red a entrenar la hicimos en base a los experimentos anteriores, con lo cual para cada uno de los tres problemas binarios entrenamos una ResNet 50 modificando sus capas \textit{fully connected} para que tengamos dos salidas en lugar de tres como anteriormente.\\

Se generaron, para evitar problemas con las etiquetas de cada clase, de modo que se estuviesen siempre en el conjunto {0,1} y ahora simplemente entrenamos cada una de las redes sobre cada uno de los conjunto creados, nuevamente con 10 \textit{epochs} de \textit{transfer learning} y otras 50 de \textit{fine tunning}. Una vez se han entrenado cada una de las redes pasamos al esquema de agregación de los resultados, las probabilidades, obtenidas para cada una de las clases. Señalar que para estos esquemas se ha supuesto que cada uno de las redes, para aquella clase que ignoran que existe, dan como probabilidad el valor 0.\\

El primer esquema que se nos ocurrió fue el más simple de todos, simplemente para cada clase la probabilidad asignada será \emp{la media} de la probabilidad para dicha clase asignada por cada una de las tres redes neuronales. No obstante este esquema tiene algo que no nos gusta y es que si por ejemplo tenemos las siguientes probabilidades:

\begin{table}[H]
\centering
\caption{Ejemplo de probabilidades}
\label{my-label}
\begin{tabular}{|c|c|c|c|}
\hline
Clasificador \textbackslash Clase & Tipo 1 & Tipo 2 & Tipo 3 \\ \hline
1-2                               & 1      & 0      & 0      \\ \hline
1-3                               & 1      & 0      & 0      \\ \hline
2-3                               & 0      & 0.5    & 0.5    \\ \hline
\end{tabular}
\end{table}

Entonces a esta imagen, usando como esquema de agregación la media, tenemos las siguientes probabilidades:

\begin{table}[H]
\centering
\caption{Probabilidades de la media}
\begin{tabular}{|c|c|c|}
\hline
Tipo 1 & Tipo 2 & Tipo 3 \\ \hline
0.67   & 0.17   & 0.17   \\ \hline
\end{tabular}
\end{table}

Como vemos mientras que dos clasificadores nos dan una confianza del 100\% en que la imagen es del Tipo 1, como el clasificador 2-3 da como probabilidad para esa clase el 0, puesto que no conoce el Tipo 1, y aunque ese clasificador no tiene certidumbre sobre la pertenencia de la imagen al Tipo 2 o al Tipo 3, dando probabilidad 0.5 para ambas, lo que ocurre es que la probabilidad que le damos a la imagen para el Tipo 1 es de 0.67. Esto nos parecía algo injusto y nos gustaría un esquema que tuviese en cuenta este tipo de escenarios, dando mayor peso a los dos primeros clasificadores que al que no está seguro sobre la clase a la que pertenece la imagen.\\

Por lo que acabamos de exponer decidimos probar con otro esquema, para ello leímos un artículo, en el que participa como autor el profesor Francisco Herrera y que referenciamos en la bibliografía de esta práctica, en el que se describen distintos esquemas de agregación para binarización de problemas OVO y OVA. Entre todos los esquemas expuestos nos decantamos por el llamado \emp{LVPC}, el cual se dice en el artículo que, en el caso en estar ante un problema con clasificadores binarios normalizados, es decir, clasificadores que para una clase den la probabilidad $\alpha$ y para la otra clase den su opuesto, $1 - \alpha$, entonces estamos antes un esquema establece una votación ponderada que penaliza a los clasificadores que no tienen confianza en su elección respecto a la pertenencia de la instancia a una de las dos clases que distingue.\\

Este esquema funciona como sigue, si llamamos $r_{ij}$ a la probabilidad que da el clasificador binario para las clases $i$ y $j$ para la pertenencia de una instancia a la clase $i$, y $r_{ji}$ a su análogo para la clase $j$. Entonces se calculan los siguientes términos:

\begin{center}
$P_{ij} = r_{ij} - min\lbrace r_{ij}, r_{ji} \rbrace$\\
$P_{ji} = r_{ji} - min \lbrace r_{ij}, r_{ji} \rbrace$\\
$C_{ij} = min\lbrace r_{ij}, r_{ji} \rbrace$\\
$I_{ij} = 1 - max\lbrace r_{ij}, r_{ji} \rbrace$
\end{center}

donde, como explica el artículo, $C_{ij}$ es el grado de conflicto (el grado en el que ambas clases son escogidas por el clasificador), $I_{ij}$ es el grado de ignorancia que tiene el clasificador (el grado en el que ninguna de las dos clases es escogida por el clasificador) y finalmente $P_{ij}$ y $P_{ji}$ es el grado de preferencia del clasificador hacia cada una de las dos clases. Entonces una vez que hemos calculado esto para cada instancia el esquema asigna la siguiente clase a esta instancia:

\begin{center}
$\underset{i = 1,...,m}{argmax} \underset{1 \leq j \neq i \leq m}{\sum} P_{ij} + \frac{1}{2} C_{ij} + \frac{N_i}{N_i + N_j}I_{ij}$
\end{center}

Donde $N_i$ es el número de instancias en nuestro conjunto de entrenamiento de la clase $i$. En nuestro caso, como realizamos un \textit{data augmentation online} entonces hemos supuesto que se mantienen las proporciones del conjunto de entrenamiento original, las del conjunto inicial y las adicionales para cada clase. Por otro lado como no queremos dar un resultado absoluto, es decir, dar probabilidad 1 a una clase y 0 al resto, lo que hemos hecho es en lugar de tomar el $argmax$ de las 3 sumatorias que se calculan es usar directamente el resultado de las sumatorias, dividido por 3, ya que nos encontramos con que el valor de las tres sumatorias sumaban 3 siempre, con lo que hemos dividido por tres para que sumen 1. Sin embargo los resultados obtenidos con este método para el mismo ejemplo anterior son los siguientes:

\begin{table}[H]
\centering
\caption{Probabilidades de LVPC}
\begin{tabular}{|c|c|c|}
\hline
Tipo 1 & Tipo 2 & Tipo 3 \\ \hline
0.67   & 0.19   & 0.14   \\ \hline
\end{tabular}
\end{table}

Por lo que no se soluciona el problema que queríamos evitar, seguimos teniendo la misma probabilidad para el Tipo 1, aunque sí que se penaliza al Tipo 3 en favor del Tipo 2 al haber más instancias de entrenamiento de este. Veremos adelante que además los resultados obtenidos con este método fueron peores que usando la media como esquema de agregación.

\subsection{Extracción de características con una CNN}

En esta ocasión el proceso fue bastante más sencillo, simplemente cargamos el modelo entrenado con la red ResNet 50, que fue el que mejor resultado nos dio de cara a la clasificación, gracias a las utilizadades de \code{Keras} que nos permiten guardar tanto la topología de la red como sus pesos, de este modo en cualquier momento se puede recuperar un modelo sin necesidad de volver a entrenarlo. Para ello cargamos el modelo con las herramientas \code{model\_from\_json} y \code{load\_weights}, y a continuación creamos un nuevo modelo a partir de este especificando que queremos como capa de salida, el parámetro \code{outputs} de la clase \code{Model}, la última de la parte convolucional de la red. Para saber qué capa escogíamos listamos los nombres de las capas de la red e identificamos el patrón de la estructura de la red y escogimos la última de convolución. Luego simplemente tuvimos que hacer una \textit{predicción} con este modelo sobre las imágenes de \textit{train} y de \textit{test} para luego usar estas características con los algoritmos que comentamos a continuación. Señalar que por falta de recurso y de tiempo en esta ocasión no se realizó \textit{data augmentation}.\\

Pasamos ahora a hablar de los dos algoritmos de clasificación que empleamos sobre este conjunto de características, 2048 para cada imagen, \emp{SVM y GBoost} ambos usados a través de la librería \code{scikit-learn} para Python. Hay que destacar que aunque esta librería no cuenta con soporte para GPU en general hemos quedado muy sorprendidos y satisfechos con el rendimiento de la implementación de estos algoritmos y la comodidad de uso que tiene toda su funcionalidad, además de la excelente documentación.

\subsubsection{SVM}

Para hacer esto creamos un modelo de predicción de tipo SVM con la función \code{SVC} de la librería \code{scikit-learn}, eligiendo que el esquema que queremos que siga para la clasificación es de tipo OVO y especificando que queremos poder obtener las probabilidades con las que ha clasificado cada una de las instancias que queremos. A continuación, y a fin de mejorar la predicción obtenida hemos creado un grid de parámetros, que se especifica usando simplemente una estructura de tipo diccionario de Python, en este grid especificamos los siguiente parámetros con los que probar:

\begin{itemize}
\item El kernel, es decir, el tipo de funciones que se usan para \textit{separar} las instancias, podrá ser o de tipo polinomial o funciones de base radial.
\item El parámetro de error C que nos indica si queremos que el hiperplano se ajuste más a los punto de \textit{training}, es decir, queremos que no se comentan muchos errores de clasificación en el train, cuando le damos a C un valor alto o si queremos un hiperplano con más margen pero que por contra cometa más errores, con una valor de C alto. En nuestro caso probamos con los valores 1 y 10.
\item La tolerancia para el criterio de parada puede ser o 0.1 o 0.001.
\end{itemize}

Entonces una vez que especificamos este grid empleamos la función \code{GridSearchCV} que buscará la mejor combinación de parámetros posible haciendo uso de un procedimiento de validación cruzada, incluso se nos da la opción de lanzar varios procesos en paralelo con distintas combinaciones de parámetros de modo que se reduzca el tiempo de cómputo. Tras este proceso por el que se seleccionan como kernel las funciones de base radial, como tolerancia el valor 0.1 y se usa como parámetro de error 10, es decir, se ajustan las funciones de base radial a los datos de entrenamiento, pasamos a entrenar el modelo con la función \code{fit} y a obtener las probabilidades de predicción sobre el conjunto de test con la función \code{predict\_proba}.

\subsubsection{GBoost}

El proceso para GBoost es muy similar sólo que ahora hacemos uso de la clase \code{GradientBoostingClassifier} y el grid de parámetros que configuramos es el siguiente:

\begin{itemize}
\item El número de estimadores puede ser 50, 100 o 200.
\item Las profundidad máxima de los árboles puede ser 3 o 5.
\item La tasa de aprendizaje podrá tomar los valores 0.001, 0.1 o 1.
\end{itemize}

Tras el proceso descrito con anterioridad se llega a que el mejor conjunto de parámetros es el TODO y nuevamente obtenemos las probabilidades con estos parámetros.

\subsection{Extracción de características con técnicas convencionales}

\subsubsection{HOG}

\subsubsection{ORB}
