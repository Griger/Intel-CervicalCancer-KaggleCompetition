\section{Introducción}

En esta práctica se va a trabajar con el \textit{dataset} de Intel \& MobileODT Cervical Cancer Screening. Este \textit{dataset} está formado por un conjunto de imágenes de úteros con cáncer cervical. Contiene imágenes de 3 tipos distintos de cáncer, estos tipos son llamados: \emp{tipo 1, tipo 2 y tipo 3}. Por lo que nos encontramos ante un problema de clasificación de multiclase, a diferencia de la práctica anterior, en la cual solo debíamos predecir si el pasajero moría o sobrevivía. El objetivo es predecir, dada una imagen, que tipo de cáncer presenta.\\

Para llevar a cabo este objetivo se hará uso de distintas técnicas del campo del \textit{deep learning}, entrenando para ello distintas redes neuronales y analizando posteriormente sus resultados. El objetivo último de esta práctica no es obtener la mejor puntuación en el ranking sino aprender e interiorizar la metodología del uso de las redes neuronales como modelo de predicción en un problema multiclase.\\

En las secciones siguientes analizaremos el proceso llevado a cabo para realizar este análisis así como las distintas herramientas y técnicas empleadas en el mismo.\\

Señalar que los experimentos se han realizado en un computador que despone de una tarjeta gráfica \textbf{NVIDIA GTX 960} con 1024 \textit{cuda cores}, un procesador \textbf{Intel Core i5 2320} y \textbf{12 GB} de memoria RAM. Para realizar las prácticas se ha hecho uso de \code{Python3} sobre todo la liberías \code{Keras}, que a su vez utiliza como \textit{backend} \code{Tensor Flow}.

